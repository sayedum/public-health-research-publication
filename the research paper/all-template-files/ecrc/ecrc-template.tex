
% Template for Elsevier CRC journal article
% version 1.1 dated 16 March 2010

% This file (c) 2010 Elsevier Ltd.  Modifications may be freely made,
% provided the edited file is saved under a different name

% This file contains modifications for Procedia Computer Science
% but may easily be adapted to other journals

% Changes since version 1.0
% - elsarticle class option changed from 1p to 3p (to better reflect CRC layout)

%-----------------------------------------------------------------------------------

%% This template uses the elsarticle.cls document class and the extension package ecrc.sty
%% For full documentation on usage of elsarticle.cls, consult the documentation "elsdoc.pdf"
%% Further resources available at http://www.elsevier.com/latex

%-----------------------------------------------------------------------------------

%%%%%%%%%%%%%%%%%%%%%%%%%%%%%%%%%%%%%%%%%%%%%%
%%%%%%%%%%%%%%%%%%%%%%%%%%%%%%%%%%%%%%%%%%%%%%
%%                                          %%
%% Important note on usage                  %%
%% -----------------------                  %%
%% This file must be compiled with PDFLaTeX %%
%% Using standard LaTeX will not work!      %%
%%                                          %%
%%%%%%%%%%%%%%%%%%%%%%%%%%%%%%%%%%%%%%%%%%%%%%
%%%%%%%%%%%%%%%%%%%%%%%%%%%%%%%%%%%%%%%%%%%%%%

%% The '3p' and 'times' class options of elsarticle are used for Elsevier CRC
\documentclass[3p,times, twocolumn]{elsarticle}

%% The `ecrc' package must be called to make the CRC functionality available
\usepackage{ecrc}

%% The ecrc package defines commands needed for running heads and logos.
%% For running heads, you can set the journal name, the volume, the starting page and the authors

%% set the volume if you know. Otherwise `00'
\volume{00}

%% set the starting page if not 1
\firstpage{1}

%% Give the name of the journal
\journalname{Procedia Computer Science}

%% Give the author list to appear in the running head
%% Example \runauth{C.V. Radhakrishnan et al.}
\runauth{}

%% The choice of journal logo is determined by the \jid and \jnltitlelogo commands.
%% A user-supplied logo with the name <\jid>logo.pdf will be inserted if present.
%% e.g. if \jid{yspmi} the system will look for a file yspmilogo.pdf
%% Otherwise the content of \jnltitlelogo will be set between horizontal lines as a default logo

%% Give the abbreviation of the Journal.
\jid{procs}

%% Give a short journal name for the dummy logo (if needed)
\jnltitlelogo{Procedia Computer Science}

%% Hereafter the template follows `elsarticle'.
%% For more details see the existing template files elsarticle-template-harv.tex and elsarticle-template-num.tex.

%% Elsevier CRC generally uses a numbered reference style
%% For this, the conventions of elsarticle-template-num.tex should be followed (included below)
%% If using BibTeX, use the style file elsarticle-num.bst

%% End of ecrc-specific commands
%%%%%%%%%%%%%%%%%%%%%%%%%%%%%%%%%%%%%%%%%%%%%%%%%%%%%%%%%%%%%%%%%%%%%%%%%%

%% The amssymb package provides various useful mathematical symbols
\usepackage{amssymb}
%% The amsthm package provides extended theorem environments
%% \usepackage{amsthm}

%% The lineno packages adds line numbers. Start line numbering with
%% \begin{linenumbers}, end it with \end{linenumbers}. Or switch it on
%% for the whole article with \linenumbers after \end{frontmatter}.
%% \usepackage{lineno}

%% natbib.sty is loaded by default. However, natbib options can be
%% provided with \biboptions{...} command. Following options are
%% valid:

%%   round  -  round parentheses are used (default)
%%   square -  square brackets are used   [option]
%%   curly  -  curly braces are used      {option}
%%   angle  -  angle brackets are used    <option>
%%   semicolon  -  multiple citations separated by semi-colon
%%   colon  - same as semicolon, an earlier confusion
%%   comma  -  separated by comma
%%   numbers-  selects numerical citations
%%   super  -  numerical citations as superscripts
%%   sort   -  sorts multiple citations according to order in ref. list
%%   sort&compress   -  like sort, but also compresses numerical citations
%%   compress - compresses without sorting
%%
%% \biboptions{comma,round}

% \biboptions{}

% if you have landscape tables
\usepackage[figuresright]{rotating}

% put your own definitions here:
%   \newcommand{\cZ}{\cal{Z}}
%   \newtheorem{def}{Definition}[section]
%   ...

% add words to TeX's hyphenation exception list
%\hyphenation{author another created financial paper re-commend-ed Post-Script}

% declarations for front matter

\begin{document}

\begin{frontmatter}

%% Title, authors and addresses

%% use the tnoteref command within \title for footnotes;
%% use the tnotetext command for the associated footnote;
%% use the fnref command within \author or \address for footnotes;
%% use the fntext command for the associated footnote;
%% use the corref command within \author for corresponding author footnotes;
%% use the cortext command for the associated footnote;
%% use the ead command for the email address,
%% and the form \ead[url] for the home page:
%%
%% \title{Title\tnoteref{label1}}
%% \tnotetext[label1]{}
%% \author{Name\corref{cor1}\fnref{label2}}
%% \ead{email address}
%% \ead[url]{home page}
%% \fntext[label2]{}
%% \cortext[cor1]{}
%% \address{Address\fnref{label3}}
%% \fntext[label3]{}

\dochead{}
%% Use \dochead if there is an article header, e.g. \dochead{Short communication}

\title{}

%% use optional labels to link authors explicitly to addresses:
%% \author[label1,label2]{<author name>}
%% \address[label1]{<address>}
%% \address[label2]{<address>}

\author{}

\address{}

\begin{abstract}
%% Text of abstract
Chronic Kidney Disease (CKD) leading to End-Stage Renal Disease (ESRD) is very prevalent today. Over 37 millions of Americans have CKD. CKD/ESRD and interrelated diseases cause a majority of the early deaths.  Many research studies have investigated the effects of drugs on CKD. However, less attention has been given to the study of the dietary patterns on CKD progression and mortality. Additionally, recent dietary recommendations shift is not extensively studied for impact on the CKD patients. This research study has uncovered significant correlations between dietary patterns and CKD mortality, also on CKD diagnostic markers such as the Albumin to Creatinine Ratio (ACR). This study also compared the findings with Dietary Recommendations shift for Impact on CKD patients. In this study, Dietary surveys from NHANES, and CKD Mortality dataset from USRDS, Food Grouping datasets from USDA, Shift Recommendation study by CDC were utilized. Principal Component Analysis and Regression were utilized to find the effect on CKD mortality. Grains, Fruits, Alcohols, Nuts showed negative correlations where Vegetables such as Other Vegetables, Starchy Vegetables, and Red and Orange Vegetables showed positive correlations. ACR values were not found strongly correlated with dietary patterns. Comparison with Dietary Recommendations Shift study found that recommended shifts on Fruits, Fats, Polyunsaturated Fats will be beneficial to CKD patients whereas adaptations to reduce harmful effects are required for Vegetables, Whole and Refined Grains.
\end{abstract}

\begin{keyword}
%% keywords here, in the form: keyword \sep keyword
%% MSC codes here, in the form: \MSC code \sep code
%% or \MSC[2008] code \sep code (2000 is the default)
hello \sep world
\end{keyword}

\end{frontmatter}

%%
%% Start line numbering here if you want
%%
% \linenumbers

%% main text
%\section{}
%\label{}
\break
\medskip
\vspace{1cm}
\noindent  Chronic kidney disease (CKD) is very prevalent in today’s world and CKD incidents are continually increasing whereby 10 to 13\% of the US population are affected by Chronic Kidney Disease\cite{Jaimonetal2017}. CKD/ESRD and other interrelated diseases such as Hypertension, Heart Diseases, and Diabetes cause the majority of early  deaths \cite{Hannan2019}. In addition to kidney failure, CKD is also a major cause of death from stroke, and heart diseases. On the other hand, hypertension and diabetes are also major causes of CKD. CKD is not reversible, progressive and gradually reduces kidney function. 

\medskip

\noindent Primary causes of CKD include High Blood Pressure, and Diabetes. Other causes include infection, kidney stones, genetics, genetical polycystic kidney disease, certain foods and food habits, pain killers, and drug usage/abuse. As CKDs are neither curable nor reversible controlling diabetes and blood pressure with or without medication can slow the progress of CKDs in most cases. As Kidneys filter waste products and our diet produces those waste products controlling diet have an effect on how much work kidney has to perform, and how well kidneys will function. Studies show that drugs as well as lifestyle choices (food, diet, exercise) can prevent CKD, slow the progression of CKD \cite{Jaceketal2017}, delay dialysis and kidney transplantation; consequently can prevent early deaths. Although there are many studies on the effect of drugs to control CKD and related complications, there are few studies on the effect of diets, dietary patterns, and lifestyles\cite{Jaceketal2017}. There  are studies on how controlling nutrients/chemicals in food items can help to prevent or to slow the progression of CKD. CKD patients are also provided recommendations on certain chemicals or food items. However, adhering to the recommended amount of nutrients and/or food item is challenging. Hence, there is an emerging trend where the effect is studied utilizing  dietary patterns with food groups and food subgroups rather than nutrients/chemicals in food or individual food item. This research analyzes the effect of dietary patterns, using food groups and food subgroups, on  the mortality and survival of CKD patients. Also studied in this research is the correlation between dietary patterns and ACR. 

As Dietary recommendations are difficult to adhere to in food item or nutrient level, there are studies such as by CDC (health.gov) to shift the recommendations in dietary patterns using food groups and subgroups. This study also has explored the dietary recommendations shift by CDC and compared the correlation of Food Groups/Subgroups with CKD Mortality and ACR to understand how the shift recommedation affect CKD patients.

\subsection{How CKD is Identified and Measured}
CKD is identified with one of two measures such as a blood test named Glomerular Filtration Rate (GFR) or a urine test named Albumin to Creatinine Ratio (ACR).  ACR values less than 30 indicates no CKD or mild CKD. ACR values between 30 and 300 indicate moderate CKD. ACR values \textgreater 300 indicates severe CKD. Patients are diagnosed with a CKD disease if the ACR values persist within the above ranges for three months. GFR is measured in ml/min/1.73 m2. CKDs as measured with GFRs are described in stages such as Stage 1 with normal or high GFR (GFR greater than or equal to  90 mL/min), Stage 2 with Mild CKD (GFR = 60-89 mL/min), Stage 3A with Moderate CKD (GFR = 45-59 mL/min), Stage 3B with Moderate CKD (GFR = 30-44 mL/min), Stage 4 with Severe CKD (GFR = 15-29 mL/min), Stage 5 with End Stage CKD (GFR \textless 15 mL/min) \cite{Huangetal2013}. At stage 5, patients loose complete kidney function. At this point, patient survival will require either dialysis or organ transplantation.

%% The Appendices part is started with the command \appendix;
%% appendix sections are then done as normal sections
%% \appendix

\section{Introduction}
Chronic kidney disease (CKD) is very prevalent in today’s world and CKD incidents are continually increasing whereby 10 to 13\% of the US population are affected by Chronic Kidney Disease. CKD/ESRD and other interrelated diseases such as Hypertension, Heart Diseases, and Diabetes cause the majority of early  deaths [31]. In addition to kidney failure, CKD is also a major cause of death from stroke, and heart diseases. On the other hand, hypertension and diabetes are also major causes of CKD. CKD is not reversible, progressive and gradually reduces kidney function. 

%% \section{}
%% \label{}

%% References
%%
%% Following citation commands can be used in the body text:
%% Usage of \cite is as follows:
%%   \cite{key}         ==>>  [#]
%%   \cite[chap. 2]{key} ==>> [#, chap. 2]
%%

%% References with BibTeX database:

\bibliographystyle{elsarticle-num}
\bibliography{<your-bib-database>}

%% Authors are advised to use a BibTeX database file for their reference list.
%% The provided style file elsarticle-num.bst formats references in the required Procedia style

%% For references without a BibTeX database:

% \begin{thebibliography}{00}

%% \bibitem must have the following form:
%%   \bibitem{key}...
%%

% \bibitem{}

% \end{thebibliography}

\end{document}

%%
%% End of file `ecrc-template.tex'. 