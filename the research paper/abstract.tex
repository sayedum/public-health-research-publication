Chronic Kidney Disease (CKD) leading to End-Stage Renal Disease (ESRD) is very prevalent today. Over 37 millions of Americans have CKD. CKD/ESRD and interrelated diseases cause a majority of the early deaths.  Many research studies have investigated the effects of drugs on CKD. However, less attention has been given to the study of the dietary patterns on CKD. Additionally, recent dietary recommendations shift is not extensively studied for impact on CKD patients. This research study has uncovered significant correlations between dietary patterns and CKD mortality, also between  dietary patterns  and CKD diagnostic markers such as the Albumin to Creatinine Ratio (ACR). This study also compared the findings with Dietary Recommendations shift for Impact Analysis on CKD patients. In this project, Dietary surveys from NHANES, and CKD Mortality dataset from USRDS, Food Grouping datasets from USDA, Shift Recommendation study from CDC were utilized. Principal Component Analysis and Regression were utilized to find the effect on CKD mortality. Grains, Fruits, Alcohols showed negative correlations where Vegetables, Other Vegetables showed positive correlations. ACR values were not found strongly correlated with dietary patterns. These finding are compared with Dietary Recommendations Shift.

%\medskip
%\noindent \textbf{Keywords:} 
%\noindent CKD, ESRD, Dietary Patterns, Mortality, ACR, Dietary Recommendations Shift