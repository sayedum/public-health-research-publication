%\thispagestyle{plain}
\begin{center} \section*{Abstract} \end{center}
Chronic Kidney Disease (CKD) leading to End-Stage Renal Disease (ESRD) is very prevalent today. Over 37 millions of Americans have CKD. CKD/ESRD and interrelated diseases cause a majority of the early deaths.  Many research studies have investigated the effects of drugs on CKD. However, less attention has been given to the study of the dietary patterns on CKD. This research study has uncovered significant correlations between dietary patterns and CKD mortality as well as identified diagnostic markers for CKD such as the Albumin to Creatinine Ratio (ACR). In this project, Dietary surveys from NHANES, and CKD Mortality dataset from USRDS were utilized to study the correlation between dietary patterns and morbidity of CKD patients. Principal Component Analysis and Regression were utilized to find the effect. Machine Learning Approaches including Regression, and Bayesian were applied to predict ACR values. Grains, Other Vegetables showed positive correlations with Mortality whereas Alcohol, Sugar, and Nuts showed negative correlations. ACR values were not found strongly correlated with dietary patterns. For ACR value prediction, 10 Fold Cross Validations with Polynomial Regression showed 95\% accuracy.

\medskip
\noindent \textbf{Keywords:} 

\noindent CKD, ESRD, Dietary Patterns, Mortality, ACR, Polynomial Regression, Bayesian