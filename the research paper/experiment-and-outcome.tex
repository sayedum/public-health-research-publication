\newcommand{\specialcell}[2][c]{%
\begin{tabular}[#1]{@{}l@{}}#2\end{tabular}}
\section{Experiment Design}
\noindent Experiments as provided in Tables \ref{experiment-start} to \ref{experiment-end} are designed to find associations between dietary patterns and CKD mortality as well as to predict ACR values. All of these experiments except set 8 and 9 are conducted.

\begin{center}
\begin{table}[!htb]
\small
\caption{\textbf{Set 1: Association of Food Groups with CKD Mortality}}
\label{experiment-start}
\vspace{0.25cm}
\begin{tabular}{| p{3cm}  |  p {12cm} | }
\hline
\noindent \textbf{Primary Input Dataset:} &  NHANES dietary intake survey aggregated by USRDS age groups to calculate average food item  intake by the participants \\
\hline
\noindent \textbf{Target Variable:} & ESRD: Avg. Annual Mortality rates \\
\hline
\noindent \textbf{Experiment 1.1:}  &   \noindent Identify contributing and important food groups in the dataset  using PCA  a) Using Actual Intake Amount  b) Using ratios of intake and recommended high amounts\\
\hline
\noindent \textbf{Experiment 1.2:}  &  Find correlation (using Pearson’s correlation and regression)  between CKD  mortality and important food groups as found  using PCA  in  experiment 1.1.   a) Using Actual Intake Amount   b) Using ratios of  intake and recommended high amounts \\
\hline
\end{tabular}
\end{table}
\end{center}

\begin{table}[!h]
\caption{\textbf{Set 2: Association of Food Subgroups with CKD Mortality}}
\label{experiment-3}
\vspace{0.25cm}
\begin{tabular}{| p{3cm}  |  p {12cm} | }
\hline
\noindent \textbf{Primary Input Dataset:} &  { NHANES survey aggregated by USRDS  age groups to calculate average food item (food subgroups) intake by the participants  }  \\
\hline
\noindent \textbf{Target Variable:} & ESRD: Avg. Annual Mortality rates \\
\hline
\noindent \textbf{Experiment 2.1:}  & {Identify important food sub groups in the dataset using PCA  and Actual  Average Intake Amount } \\
\hline
\noindent \textbf{Experiment 2.2:}  & {Similar to experiment 1.2 (Regression); however, used food  subgroups  and actual average intake amount only} \\
\hline
\end{tabular}
\end{table}

\begin{table}[!htb]
\caption{\textbf{Set 3: Effect of Food Groups on ACR}}
\label{experiment-4}
\vspace{0.25cm}
\begin{tabular}{| p{3cm}  |  p {12cm} | }
\hline
\noindent \textbf{Primary Input Dataset:} & NHANES survey data for each participant; intake amounts for food groups were averaged for  two surveys. This data was merged with  laboratory tests for ACR \\
\hline
\noindent \textbf{Target Variable:} & Albumin to Creatinine Ratio (ACR) \\
\hline
\noindent \textbf{Experiment 3.1:}  & { Identify contributing and important food groups in the input dataset using PCA.  This is  different  than Experiment 1 because entire survey is  being used here; not the  aggregated  data by age groups} \\
\hline
\noindent \textbf{Experiment 3.2:}  & { Find out correlation (using Pearson’s correlation, and regression)  between ACR  Values and important food groups as found using  PCA in experiment 3.1.}  \\
\hline
\end{tabular}
\end{table}

\begin{table}[!ht]
\caption{\textbf{Set 4: Effect of Nutrients on ACR}}
\label{experiment-5}
\vspace{0.25cm}
\begin{tabular}{| p{3cm}  |  p {12cm} | }
\hline
\multicolumn{2}{|p{15cm}|}  { Utilize the same experiments as done for ACR and Food Groups. However, use nutrients intake a) with  or b) without  combining with food groups } \\
\hline
\noindent \textbf{Experiment 4.1:} & PCA to identify contributing factors \\
\hline
\noindent \textbf{Experiment 4.2:} &Regression to find correlations among factors found in experiment 4.1 \\
\hline
\end{tabular}
\end{table}

\begin{table}[!ht]
\small
\caption{\textbf{Set 5: Effect of Food Subgroups on ACR}}
\label{experiment-6}
\vspace{0.25cm}
\begin{tabular}{| p{3cm}  |  p {12cm} | }
\hline
\noindent \textbf{Primary Input Dataset:} & { NHANES survey data for each participant; intake amounts for food subgroups were averaged for  two surveys. This data was merged with  laboratory tests for ACR } \\
\hline
\noindent \textbf{Target Variable:} & Albumin to Creatinine Ratio (ACR) \\
\hline
\multicolumn{2}{|c|} { {Similar experiments like set 3 and set 4. However, used food subgroups as the input/source  variables}} \\
\hline
\end{tabular}
\end{table}

\begin{longtable}{| p{3cm}  |  p {12cm} | } 
\caption{\textbf{Set 6: Use Regression and Bayesian to predict ACR using Food Subgroups intake}} \\
\hline \multicolumn{2}{|p{15cm}|} {  { \noindent Utilize Machine Learning Approaches for ACR prediction in test dataset.  }} \label{experiment-7}  \\ \hline
\noindent \textbf{Primary Input Dataset:} & { Input dataset  from Set 5 is used here as the input dataset } \\ \hline
\noindent \textbf{Target Variable:} & ACR Values; Also, ACR categories (CKD or Not) \\ \hline
\noindent \textbf{Experiment 6.1:} &   {ACR value prediction using linear regression. (ACR category was also  an option) }  \\ \hline
\noindent \textbf{Experiment 6.2:} &  Conduct experiment 6.1; Use 10 folds cross validation where applicable.\\ \hline
\noindent \textbf{Goal:} & { Find  ACR predictability in the test dataset such as calculate R2 Score or generate Confusion Matrix} \\ \hline
\noindent \textbf{Experiment 6.3:} & Conduct experiment 6.1; however use Polynomial Regression \\ \hline
\noindent \textbf{Experiment 6.4:} &  Conduct experiment 6.3; With 10 Folds Cross Validations  \\ \hline
\noindent \textbf{Experiment 6.5:} & { Conduct experiment 6.1; however, use a) Random Forest Regression  with or  without 10 Folds  Cross Validations. b) Utilize  Polynomial Regression in  the process.} \\ \hline
\noindent \textbf{Experiment 6.6:} & { Conduct experiment 6.1; however, use a) Bayesian prediction with or without  10 Folds  Cross Validations.  b) Use Polynomial Fit} \\ \hline
\end{longtable}

\begin{table}[!ht]
\caption{\textbf{Set 7: Find effect on  CKD Mortality using survey data with no data aggregation by Age Groups}}
\label{experiment-8}
\vspace{0.25cm}
\begin{tabular}{| p{3cm}  |  p {12cm} | }
\hline
\noindent \textbf{Input Data:} & Bring CKD mortality data to each participant using the corresponding ages\\
\hline
\multicolumn{2}{|p{15cm}|} { { \noindent Use PCA (to find contributing food groups and subgroups) and then Regression to find  correlation  between mortality and Food Groups/Subgroups/ACR values}} \\
\hline
\end{tabular}
\end{table}

\begin{table}[!ht]
\caption{\textbf{Set 8:  Utilize Regression and Bayesian to predict Mortality  using survey data with no data aggregation by Age Groups}}
\label{experiment-9}
\vspace{0.25cm}
\begin{tabular}{| p{3cm}  |  p {12cm} | }
\hline
\noindent \textbf{Input Data:} & Bring CKD mortality data to each participant using the corresponding age i.e. Input dataset from Set 7 can be used here as the Input dataset \\
\hline
\multicolumn{2}{|p{15cm}|} { { \noindent  And then utilize Machine Learning Approaches including Regression and Bayesian for Mortality Prediction on Test Dataset.   }} \\
\hline
\end{tabular}
\end{table}

\begin{table}[!ht]
\caption{\textbf{Set 9: Association between Food Groups/Subgroups and Remaining Life for CKD Patients: Use not aggregated dietary intake data}}
\label{experiment-10}
\label{experiment-end}
\vspace{0.25cm}
\begin{tabular}{| p{3cm}  |  p {12cm} | }
\hline
\noindent \textbf{Input Data:} & {Bring remaining life data such as 5 years survival probabilities for each participant using the corresponding age and CKD status } \\
\hline
\multicolumn{2}{|p{15cm}|} { Use PCA (to find contributing food groups and subgroups) and then utilize Regression to find correlation between remaining life probabilities and Food Groups/Subgroups} \\
\hline
\end{tabular}
\end{table}