Chronic Kidney Disease (CKD) leading to End-Stage Renal Disease (ESRD) is very prevalent today. Over 37 millions of Americans have CKD. CKD/ESRD and interrelated diseases cause a majority of the early deaths.  Many research studies have investigated the effects of drugs on CKD. However, less attention has been given to the study of the dietary patterns on CKD progression and mortality. Additionally, recent dietary recommendations shift is not extensively studied for impact on the CKD patients. This research study has uncovered significant correlations between dietary patterns and CKD mortality, also on CKD diagnostic markers such as the Albumin to Creatinine Ratio (ACR). This study also compared the findings with Dietary Recommendations shift for Impact on CKD patients. In this study, Dietary surveys from NHANES, and CKD Mortality dataset from USRDS, Food Grouping datasets from USDA, Shift Recommendation study by CDC were utilized. Principal Component Analysis and Regression were utilized to find the effect on CKD mortality. Grains, Fruits, Alcohols, Nuts showed negative correlations where Vegetables such as Other Vegetables, Starchy Vegetables, and Red and Orange Vegetables showed positive correlations. ACR values were not found strongly correlated with dietary patterns. Comparison with Dietary Recommendations Shift study found that recommended shifts on Fruits, Fats, Polyunsaturated Fats will be beneficial to CKD patients whereas adaptations to reduce harmful effects are required for Vegetables, Whole and Refined Grains.