\section *{Conclusions}
CKD leading to End Stage Renal Disease is very prevalent today; treatment facilities for dialysis, and donors for organ transplantation are limited. Consequently, many patients die waiting for  proper treatment \cite{Bloomberg2019}. Majority of the studies focused on drugs to control CKD progression where some studies focused on diets, nutrients, and individual food items. Controlling CKD using changes to dietary patterns can be beneficial to both the patients and the economy. Hence, this study focused on the effect of dietary patterns on CKD mortality and a CKD measure named Albumin to Creatinine Ratio (ACR). This study also compared the findings with Dietary Recommendations shift for Impact on CKD patients. PCA and Regression are used to study the association between ACR and CKD mortality/survival. Grains, Fruits, Alcohols, Nuts showed negative correlations where Vegetables such as Other Vegetables, Starchy Vegetables, and Red and Orange Vegetables showed positive correlations. Overall, the results of this research matched the findings of other studies and current knowledge with few exceptions. ACR values were not found strongly correlated with dietary patterns. Comparison with Dietary Recommendations Shift study found that recommended shifts on Fruits, Fats, Polyunsaturated Fats will be beneficial to CKD patients whereas adaptations to reduce harmful effects are required for Vegetables, Whole and Refined Grains.