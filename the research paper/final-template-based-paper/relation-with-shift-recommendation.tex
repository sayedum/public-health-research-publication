\subsection{Relation with Shift Recommendation}

\noindent \noindent \textbf{Recommended Shift:} Include more vegetables from all subgroups where with few exceptions, the U.S. population does not meet intake recommendations for any of the vegetable subgroups. 

\noindent \textbf{Our Study:} This study found moderate positive correlations with mortality for CKD patients. Hence, more vegetables intake is from all vegetable subgroups is not highly recommended  for CKD patients. As this study also finds that food subgroups such as Other Vegetables, Red and Orange Vegetables, and Starchy vegetables are positively correlated with mortality; hence, limiting these subcategories with moderate intake of overall vegetables can be beneficial to CKD patients. 

\noindent\rule{7.8cm}{0.4pt}

\noindent \textbf{Recommended Shift:} Increase fruit (whole fruit) intake for all individuals. Also, take fruits as snacks, salads, and side-dishes.

\noindent \textbf{Our Study:} This study finds Fruits has negative correlations with CKD mortality. Hence, more fruits intake will be beneficial to CKD patients. Shift recommendation is in align with our study.

\noindent\rule{7.8cm}{0.4pt}

\noindent \textbf{Recommended Shift:} Half of all grains should be whole grain.

\noindent \textbf{Our Study:} This study found strong negative correlation of Grains with CKD mortality i.e. more Grain intake showed low mortality rates. Hence, like shift recommendations, CKD patients will benefit by taking recommended amount of Grains. However, this Grain intake includes both whole and refined grain intake. This study also found that Whole grain has moderate negative correlation with mortality i.e. Refined grain contributed to the overall strong correlation. Hence, refined grain-based products can be seen to contribute in negative correlation. Whole grains having rich in Potassium are not highly recommended for CKD patients (I have to check for what level of CKD). CKD patients might get benefit with a mix of whole and refined grain based products where finding the ideal ratio will need further research.

\noindent\rule{7.8cm}{0.4pt}

\noindent \textbf{Recommended Shift:} Increase dairy intake in fat-free form.

\noindent \textbf{Our study:}  This study did not find any strong correlation between dairy products and mortality. However, this study found positive correlation of Urine ACR with dairy products i.e. Urine ACR is high when dairy products are taken more. 

However, the recommended Shift i.e. Increase dairy intake in fat-free form consistent with the recommendation done to CKD patients.

\noindent\rule{7.8cm}{0.4pt}

\noindent \textbf{Recommended Shift:} Increase seafood intake where Teen boys and adult men are recommended to reduce protein intake (instead increase/replace that with vegetables)

\noindent \textbf{Our study:} This study did not find any strong correlation between protein and mortality. 

However, CKD patients based on stages of CKD are recommended to take moderate amount of Protein. The recommended shift i.e. Reduce protein intake for Teen Boys and Adult Men will help CKD patients for that age groups along with others. 

\noindent\rule{7.8cm}{0.4pt}

\noindent \textbf{Recommended Shift:} Increase Oil intake and reduce solid fat intake. Use Oils rather than solid fats.

\noindent \textbf{Our Study:} This study did not see any strong correlation in this regard for CKD patients

\noindent\rule{7.8cm}{0.4pt}

\noindent \textbf{Recommended Shift:} Reduce added sugars to less than 10\% of calories intake per day. 

\noindent \textbf{Our Study:} This study found negative correlation between CKD mortality and Added Sugars i.e. Less sugar intake is related to high mortality. This is contradictory to current knowledge and goes against shift recommendation. However, investigation is required on our data and finding to see what were included in the sugar category. 

\noindent\rule{7.8cm}{0.4pt}

\noindent \textbf{Recommended Shift:} Reduce saturated fat intake to 10\% of calories per day. Shift to take more  polyunsaturated and monounsaturated fats than saturated fats.

\noindent \textbf{Our study:} Polyunsaturated fatty acids have negative correlation with CKD mortality i.e. CKD patients will benefit by taking more  Polyunsaturated fatty acids. Hence, recommended shift will benefit CKD patients.

\noindent\rule{7.8cm}{0.4pt}

\noindent \textbf{Recommended Shift:} Reduce Sodium Intake.

\noindent \textbf{Our Study:} This study did not find any correlation between Sodium and CKD Mortality (I have to verify to what extent sodium is included or not in the study).
