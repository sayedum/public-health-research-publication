\subsection *{Relation with Shift Recommendation}
The findings on effect of Dietary Patterns  on CKD patients from this study are compared with Dietary Recommendations Shift study, and we identified the recommendations that will be helpful to CKD patients along with areas where CKD patients will need to adapt and modify the recommendations. In summary, the comparison shows that recommended shifts on Fruits, Fats, and Polyunsaturated Fats will be beneficial to CKD patients whereas adaptations to reduce harmful effects are required for Vegetables, Whole and Refined Grains. Below more details on the comparison are provided. 

\noindent \noindent \textbf{Recommended Shift:} Include more vegetables from all subgroups. The study found with few exceptions, the U.S. population does not meet intake recommendations for any of the vegetable subgroups.

\noindent \textbf{This Study:} This study found moderate positive correlations with mortality for CKD patients. Hence, more vegetables intake from all vegetable subgroups is not highly recommended  for CKD patients. As this study also found that food subgroups such as Other Vegetables, Red and Orange Vegetables, and Starchy vegetables are positively (moderate, 0.45 to 0.70) correlated with mortality; hence, CKD patients can get benefit by taking limited amount of vegetables from these subgroups where can take moderate to high amount of vegetables from other subgroups.

\noindent\rule{9cm}{0.4pt}

\noindent \textbf{Recommended Shift:} Increase fruit (whole fruit) intake for all individuals. Also, take fruits as snacks, salads, and side-dishes.

\noindent \textbf{This Study:} This study finds Fruits have negative correlations with CKD mortality. Hence, more fruits intake will be beneficial to the CKD patients. Shift recommendation is in aligned with this study.

\noindent\rule{9cm}{0.4pt}

\noindent \textbf{Recommended Shift:} Half of all grains should be Whole Grain.

\noindent \textbf{This Study:} This study found strong negative correlation of Grains with CKD mortality i.e. more Grain intake showed low mortality rates. Hence, as recommended by shift recommendations, CKD patients will benefit by taking recommended amount of Grains. However, this finding on Grain intake includes both whole and refined grain intake. This study also found that Whole grain has moderate negative correlation with mortality i.e. Refined grain contributed to the overall strong correlation for Grains. Hence, refined grain-based products can be seen to contribute in negative correlation. Whole grains having rich in Potassium are not highly recommended for CKD patients. Hence, CKD patients might get benefit with a mix of whole and refined grain based products where finding the ideal ratio to take will need further research.

\noindent\rule{9cm}{0.4pt}

\noindent \textbf{Recommended Shift:} Increase dairy intake in fat-free form.

\noindent \textbf{This Study:}  This study did not find any strong correlation between dairy products and mortality. However, this study found positive correlation of Urine ACR with dairy products i.e. Urine ACR is high with increasing intake of  dairy products. However, the recommended shift i.e. increase dairy intake in fat-free form is consistent with the recommendation done to CKD patients.

\noindent\rule{9cm}{0.4pt}

\noindent \textbf{Recommended Shift:} Increase seafood intake where teen boys and adult men are recommended to reduce protein intake (instead increase/replace Protein with Vegetables)

\noindent \textbf{This Study:} This study did not find any strong correlation between protein and mortality. However, CKD patients based on stages of CKD are recommended to take moderate amount of Protein. The recommended shift i.e. Reduce protein intake for teen boys and adult men will help CKD patients for that age groups along with others. 

\noindent\rule{9cm}{0.4pt}

\noindent \textbf{Recommended Shift:} Increase Oil intake and reduce solid fat intake. Use Oils rather than solid fats.

\noindent \textbf{This Study:} This study did not see any strong correlation in this regard for CKD patients. However, as found that Polyunsaturated fatty acids have negative correlation with CKD mortality. CKD patients can get benefit by reducing solid fats which are full of saturated fats \cite{Choosemyplate2015} and by increasing Oils rich in Polyunsaturated fatty acids. 

\noindent\rule{9cm}{0.4pt}

\noindent \textbf{Recommended Shift:} Reduce added sugars to less than 10\% of calories intake per day. 

\noindent \textbf{This Study:} This study found negative correlation between CKD mortality and Added Sugars i.e. Less sugar intake is related to high mortality. This is contradictory to current knowledge and goes against shift recommendation. However, the studies might differ in what are regarded as added sugars, and the adjustments used to measure sugar content.

\noindent\rule{9cm}{0.4pt}

\noindent \textbf{Recommended Shift:} Reduce saturated fat intake to 10\% of calories per day. Shift to take more  polyunsaturated and monounsaturated fats than saturated fats.

\noindent \textbf{This Study:} Polyunsaturated fatty acids have negative correlation with CKD mortality i.e. CKD patients will benefit by taking more  Polyunsaturated fatty acids. Hence, recommended shift will benefit CKD patients.

\noindent\rule{9cm}{0.4pt}

\noindent \textbf{Recommended Shift:} Reduce Sodium Intake.

\noindent \textbf{This Study:} This study did not find any correlation between Sodium and CKD Mortality. However, Sodium in general are recommended to take less to CKD patients.  (I have to verify to what extent sodium is included or not in the study).